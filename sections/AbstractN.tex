\begin{abstract}
OWGIS has become a well stablished open source software for the automatic generation of Web-GIS sites that
display geotemproal data, especially ocean and climate. The latest version of OWGIS
incorporates outstanding visualizations with analysis tools that allow an easier understanding
of the data. Some of the features added recently are: dynamic vertical profiles with temporal animations,
3-dimensional animations of vector fields using streamlines, earth view,  automatic
generation of mobile interfaces and APK files, and dynamic time series. Additionally, OWGIS
allows the advanced users to customize the design of the websites with custom color palettes,
multiple base maps, and complete toolbox to improve the streamline animations. 
In this work the latest version of OWGIS is described and a sucesfull use case,
the meteorological atlas of the Gulf of Mexico, is presented as an example. 
\end{abstract}

%     This work describes OWGIS, an open source Java Servlets web application that creates 
%    Web GIS sites by automatically writing HTML and JavaScript code. 
%    OWGIS is configured by XML files that define which layers (geographic datasets) will be displayed 
%    on the Web GIS sites as well as the text to be used on the generated interfaces.
%    This project uses the Web Map Service (WMS) and the Web Feature Service (WFS) standards of the 
%    Open Geospatial Consortium's to request data from typical map 
%    servers, such as GeoServer or Map Server. OWGIS also uses the WMS extension proposed 
%    by \cite{ncWMS} to request data from ncWMS servers. This allows building
%    Web GIS sites that display not only vector data and raster data but also 3D data.
%    Some of the features available on the sites built with OWGIS are: multiple languages, 
%    punctual data, transparency, animations, vertical profiles and
%    vertical transects, Contextual Query Language (CQL) filtering,
%    color palettes, color ranges, and the ability to download data in
%    different formats.
%    OWGIS main users are scientists, such as oceanographers or climatologists, 
%    who store their data in NetCDF files and want to be able to 
%    analyse, share or compare their data and results using a well designed Web GIS site.
%    A discussion of the architecture of this project, a detailed description
%    of the configuration files, and an explanation of the design priciples used on this project are presented.
%    This work describes OWGIS, an open source Java web application that creates 
%    Web GIS sites by automatically writing HTML and JavaScript code. 
%    OWGIS is configured by XML files that define which layers (geographic datasets) will be displayed 
%    on the websites.  
%    This project uses several 
%    Open Geospatial Consortium standards  to request data from typical map 
%    servers, such as GeoServer, and is also able to request
%    data from ncWMS servers. 
%     The latter allows for the displaying of 4D data stored using the NetCDF file format (widely
%    used for storing environmental model datasets). 
%    Some of the features available on the sites built with OWGIS are: multiple languages, 
%    animations, vertical profiles and vertical transects, 
%    color palettes, color ranges, and the ability to download data.
%    OWGIS main users are scientists, such as oceanographers or climate scientists, 
%    who store their data in NetCDF files and want to 
%    analyze, visualize, share, or compare their data using a website.

