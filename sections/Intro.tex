 \section{Introduction}

 %    Over the last decades the amount of geospatial data has grown rapidly
 %     as a result of the number of satellites and the use of  
 %    Global Positioning Systems (GPS). 
 %    To assist in the analysis and visualization of all this data, several
 %    geographic information systems (GISs)
 %    and Web GIS sites have been developed \citep{steiniger_free_2012}. 
 %    Some of the Web GIS sites share common functionalities, such as
 %    access to raw data or the ability to overlap different layers 
 %    \citep{kulawiak_interactive_2010, karnatak_spatial_2012, nair_web_2011}.
 %    Web GIS sites that display environmental data, like the Pacific Islands
 %    Ocean Observing System (\href{http://oos.soest.hawaii.edu/pacioos/voyager/index.html}{Voyager})
 %    or the \href{http://www.eea.europa.eu/data-and-maps}{Web GIS sites} developed
 %    by the European Environment Agency,
 %    have an even larger set of common features, such as   
 %    the ability to identify layer data at any coordinate, and the capability to plot 
 %    data in real time. However, there is not, to the best of our knowledge,
 %    any open source program that automatically generates Web GIS sites with a minimum set 
 %    of features that allow efficient publication of 4D geospatial data.\\

 %    A common procedure for publishing georeferenced data on a Web GIS site includes  
 %    the following steps. First, generate the data to be published. 
 %    Then, upload the data into a map server such as 
 %    \href{http://mapserver.org/}{MapServer} \citep{mapserver05},
 %    \href{http://geoserver.org/}{GeoServer} \citep{deoliveira_geoserver:_2008},
 %    \href{http://www.esri.com/software/arcgis/arcgisserver}{ArcMap Server},
 %    \href{http://www.resc.rdg.ac.uk/trac/ncWMS/}{ncWMS} \citep{ncWMS}, etc. 
 %     Finally, with a group of programmers and web designers, build the web
 %    interface (the Web GIS site) to display the data for the user. This
 %    interface provides several ways to visualize and serve data, and,
 %    even when most Web GIS sites do not have direct access to the data, they normally obtain 
 %    the data through map severs.  The software described in this paper greatly simplifies the 
 %    final step, building Web GIS sites, thus reducing the time and
 %    costs for publishing geospatial data on the web.

 %    Free and Open Source Software (FOSS) that helps building
 %    Web GIS sites already exists.
 %    Figure \ref{fig:foss}, adapted from \citep{steiniger_2012_2013}, shows the FOSS
 %    GIS software that is available to aid publishing geographic data on the web at
 %    any of the previously discussed steps.
 %    The software in the \emph{Desktop GIS} category are programs that
 %    run on personal computers and are able to display, query, update, and
 %    analyze geographic data. Quantum GIS \citep{quantum_2007} 
 %    and GRASS \citep{neteler_grass_2012} are two of the most 
 %    mature FOSS desktop GIS programs. These two programs can accomplish a multitude
 %    of tasks and have been used to prepare and analyze environmental data 
 %    for several projects \citep{gkatzoflias_development_2013,grinand_estimating_2013}.

 %    The \emph{Web Map Servers} category accommodates
 %    software that provides several options 
 %    to access and visualize georeferenced data through 
 %    standards of the Open Geospatial Consortium (Web Mapping Services (WMS)  
 %    \citep{de_la_beaujardiere_opengis_2006},
 %    Web Feature Services (WFS) \citep{wfs2005} , and 
 %    Web Coverage Services (WCS) \citep{baumann_ogc_2010}).
 %    The two best-known web map servers are MapServer
 %    and Geoserver, both of them offer vector and raster support
 %    and have comparable functionality to similar proprietary software 
 %     \citep{steiniger_2012_2013}. 

 %   \begin{figure}[h]
 %       \centering
 %       \includegraphics[totalheight=.48\textheight]{../images/GISSoftwareMap/GIS_soft.png}
 %       \caption{Simplified scheme of the free and open source geographic information software for the web.
 %               Adapted from \cite{steiniger_2012_2013}. }
 %      \label{fig:foss}
 %   \end{figure}
 %    Another web map server, that is worth mentioning for the 
 %    environmental modeling community, is the ncWMS which has the 
 %    particular feature of being able to serve 
 %    4D data stored as NetCDF files, 
 %    a widely used file format for storing environmental model 
 %    datasets \citep{netcdf1990}.  The DEFENSE \citep{tiranti_defense_2014}, 
 %    and the interactive visualization system by \cite{kulawiak_interactive_2010} are
 %    some examples of environmental decision support systems 
 %     that use FOSS web map servers.\\

 %    The \emph{Web GIS libraries} and the
 %    \emph{Web Map Development Frameworks} are two categories that
 %    encompass GIS software used to build the final interface to
 %    display geographic data as dynamic maps on the web. The
 %    web GIS libraries are application programming interfaces (API's)
 %    for the visualization and manipulation of spatial data that implement
 %    the WMS, WFS, etc. OpenLayers \citep{hazzard_openlayers_2011} is
 %    one of the most extensive libraries in this category, it is implemented in JavaScript.
 %    Finally, the software in the \emph{Web Map Development Frameworks} category,
 %    contribute with tools for creating advanced web applications. These frameworks use the web GIS libraries
 %    and contain tools for the management of layers, menus, and themes of the
 %    Web GIS sites. Some well known programs in this category are Mapbender3 (\url{http://mapbender3.org/}),
 %    GeoMajas (\url{http://www.geomajas.org/}), MapFish (\url{http://www.mapfish.org/})
 %    and GeoMOOSE (\url{http://www.geomoose.org/}). OWGIS belongs to this last category as it gives an easy
 %    way to develop Web GIS sites through the configuration of XML files. The main 
 %    feature that differentiates OWGIS from the rest of the development
 %    frameworks is the set of tools that OWGIS provides to visualize
 %    4D data through the web. \\

 %    OWGIS (\url{http://owgis.org}), originally \emph{Open Web GIS}, is an open source
 %    software that creates self-contained Web GIS sites with common features 
 %    used by the scientific community. OWGIS is a Java
 %    web application that generates websites 
 %    by automatically writing HTML and JavaScript code. 
 %    The websites built with OWGIS are configured by XML files and can accommodate maps with 
 %    layers served through any map server, such as 
 %    Geoserver, that complies with the WMS standard. OWGIS is also capable of constructing Web GIS sites that display 3D data 
 %    served by the THREDDS servers \citep{JoDI51}
 %    or the ncWMS servers, using the WMS extension proposed by \cite{ncWMS}.\\
 %    
 %    The current features that OWGIS provides on the generated Web GIS sites are:
 %   multiple languages, animations, mobile interface, Contextual Query Language 
 %   (CQL) filtering (\url{http://en.wikipedia.org/wiki/Contextual\_Query\_Language}),
 %   identify features, the ability to download data as KML, GeoTIFF or shape files,
 %   and the capacity to plot vertical profiles and vertical transects at different locations.  
 %   Though some of these features are already available in many Web GIS sites,
 %    regardless of whether or not they display scientific data or other types of geographic data, 
 %     OWGIS allows building and maintaining new sites with all these features 
 %     by simply editing XML files. 

 %    OWGIS is being developed at the Center for Ocean-Atmospheric Prediction Studies
 %   (COAPS), Florida State University (FSU), in collaboration with the Universidad Nacional 
 %   Aut\'onoma de M\'exico (UNAM). OWGIS is currently used as the interactive 
 %   visualization map of the Digital Climatic Atlas of Mexico \citep{zavala2010digital},
 %   which has been available since 2009 and provides access
 %   to more than 2,000 layers of oceanic climate, climate chage scenarios, bioclimatic parameters,
 %   and socioeconomic indicators, among other variables. OWGIS is also used by the Deep-C Consortium (\url{http://deep-c.org})
 %   to display oceanographic data from the Gulf of Mexico.  

 %   The paper is organized as follows. Section \ref{sec:arch} describes the
 %   architecture of OWGIS and the design principles that it follows when creating websites.
 %   Section \ref{sec:conf} explains how to configure OWGIS for new applications.  Section 
 %   \ref{sec:features} illustrates the current features provided by OWGIS.  Section 
 %   \ref{sec:usecase} describes a specific case study where 
 %   OWGIS is successfully used to display ocean data from the Gulf of Mexico. Section \ref{sec:future} 
 %   ends with a discussion on the presented software and outlines some future functionality of OWGIS. 
