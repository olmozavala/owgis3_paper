 \section{Discussion and Conclusion}
 \label{sec:future}

% OWGIS is the first open source software that can build
% Web GIS sites that display 2D, 3D, and 4D data served from distinct map servers 
% that can be located anywhere. The main features 
% OWGIS provides in the interfaces it builds are: 
% multiple languages; animations; vertical profiles and vertical transects; color palettes; and
%the ability to download data. All these features 
%are created automatically depending on the type of data 
%and without any additional web programming. \\
%
%In the introduction, OWGIS was placed inside the vast 
%family of open source programs where OWGIS's tools for
%visualizing 4D data stand out. Within the commercial
%GIS software, where esri is the international leading enterprise,
%the overview is similar. 
%Some examples of environmental Web GIS sites created with esri's system
% ArcGIS are: PRAGIS \citep{mccool_pragis:_2014},
%Virtual Fire \citep{kalabokidis_virtual_2013}, and
%River run \citep{halls_river_2003}.
% ArcGIS online (\url{http://arcgis.com}) 
% is the key component of the ArcGIS system that provides similar features
% than OWGIS to build Web GIS sites.  ArcGIS online has a web interface that allows the user
% to choose from different base maps,
% and it has a searching tool where the users can find content that 
% has been made public by the community. It also has the capability
% of displaying 4D layers on the maps it creates, as well
% as the option to download layers in GIS format or as KML files. OWGIS
% lacks an administration interface as well as a searching engine for 
% geographical data on the web. Nevertheless, 
% OWIGS exceeds ArcGIS online in its features provided to 
% display and style 4D data acquired
% from ncWMS servers. The only ArcGIS online example, that we know off,
% that has the ability to retrieve data from ncWMS servers  
% (\url{http://dtc-sci01.esri.com/MultiDimWMSViewer/}) 
% do not has the possibility to modify color palettes 
% or the color ranges of layers. This example also lacks
% the option to crate vertical profiles, vertical transects, and time series.
% These extra features, plus the fact that
% ArcGIS online license costs thousands of dollars per year, makes OWGIS
% a good alternative to build WebGIS interfaces, specially for environmental
% scientists that store their geospatial data as NetCDF files. \\
%
%Configuring new instances of OWGIS is made easy through
% XML files. In these files, the layers and texts of the  
% websites are defined, providing an easy way to add and edit
%    new layers that expand the content of each project. At the same time it allows sharing content
%    between institutions and eases the maintenance of the websites. 
%    OWGIS conveniently stores all the texts in
%    separate files to allow multiple languages on the Web GIS sites,
%    thus increasing the potential number of users who can access the data.\\
% 
% The current version of OWGIS uses Java 7 and OpenLayers 2
% to build maps.  This version has been used by 
% scientists who store their data in NetCDF files, 
% but it has also been used to display other types of geospatial
% data. It is used to visualize ocean variables of the Gulf of Mexico
% in the \emph{Deep-C MapViewer} project
% (\href{http://viewer.coaps.fsu.edu/DeepCProject/mapviewer}{http://viewer.coaps.fsu.edu/DeepCProject/mapviewer});
% it is used to display forecast data for the ocean and the atmosphere 
% using the Weather Research and Forecasting
% Model (WRF) and the Regional Ocean Modeling System (ROMS), also in the Gulf of Mexico area,
% (\url{http://viewer.coaps.fsu.edu/GoM-FS/mapviewer});
% is the main component of the Digital Climatic Atlas of Mexico,
% which displays more than 300 layers of climate data
% (\href{http://uniatmos.atmosfera.unam.mx/ACDM/servmapas}{http://uniatmos.atmosfera.unam.mx/ACDM/servmapas});
% and it has also been used to display
% public transportation data 
% (\href{http://viajandodf.com}{http://viajandodf.com})
% and useful information for immigrants
% (\href{http://www.americas.datafest.net/resources/project-list/beat-the-beast-in-the-steeplechase}{http://www.americas.datafest.net}). \\
% 
% The current development of OWGIS is focused on incorporating
% new web technologies that allow visualizing data faster.
% OWGIS 2.0 will use OpenLayers 3, taking 
% advantage of WebGL and HTML5 to build new and faster responsive Web GIS sites,
% placing emphasis in the user experience for mobile devices.
% We hope that OWGIS becomes a popular open source software and 
% the standard tool for building Web GIS sites
% in scientific fields where netCDF files are the default file format
% for storing data.
%
